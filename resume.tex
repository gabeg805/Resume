\documentclass[margin,line,a4paper]{resume}
\usepackage[hidelinks]{hyperref}
\usepackage{cmap}
\usepackage{enumitem}
%% \usepackage[top=0.75in, bottom=0.5in, left=0.75in]{geometry}
\newcommand{\about}{\raise.17ex\hbox{$\scriptstyle\mathtt{\sim}$}}

%%______________________________________________________________________________

\begin{document}
\begin{tabular*}{7in}{l@{\extracolsep{\fill}}r}
\textbf{\Large Gabriel Gonzalez}  \\
621 West Cota Street Unit B    &  \textbf{\href{mailto:gabeg@bu.edu}{gabeg@bu.edu}} \\
Santa Barbara, CA 93101    &  \textbf{Github:} \href{http://github.com/gabeg805}{github.com/gabeg805} \\
(805) 895-4626             &  \textbf{LinkedIn:} \href{http://linkedin.com/in/gabeg805}{linkedin.com/in/gabeg805}
\end{tabular*}
\vspace{2mm}
\begin{resume}
    
    
    %%__________________________________________________________________________
    %% Objective / Summary
    \section{\mysidestyle Objective}
    I am seeking a software engineering position where I can make a positive
    contribution to the aerospace industry by applying my knowledge of programming and
    astrophysics.
    
    
    %%__________________________________________________________________________
    %% Computer Skills
    \section{\mysidestyle Computer\\Skills}
    \vspace{-0.01mm}
    \begin{description}
        \item[Languages:] C, C++, Java, Python, LaTeX, Bash, Assembly, IDL
        \item[Software:] GNU Debugger (GDB), Git, Arduino IDE
        \item[Operating systems:] Unix, Windows, OS X
        \item[Courses:] Computer Systems, Data Structures and Algorithms,
          Electromagnetic Fields and Waves, Methods of
          Theoretical Physics, Quantum Physics, Statistical Thermodynamics
    \end{description}
    \vspace{-3mm}
    
    
    %%__________________________________________________________________________
    %% Education
    \section{\mysidestyle Education}
    \textbf{Boston University \hfill September 2011 -- present} \\
    \textsl{BA : Astronomy and Physics}
    \vspace{1mm}
    \begin{itemize}[leftmargin=2em]
        \item \textbf{GPA:} 3.16
        \item \textbf{Expected graduation:} May 2015
    \end{itemize}
    
    
    %%__________________________________________________________________________
    %% Education
    \section{\mysidestyle Honors}
    \textbf{Boston University} \\
    \textsl{Dean's List} \hfill \textbf{Fall 2011}
    
    
    %%__________________________________________________________________________
    %% Experience
    \section{\mysidestyle Experience}

    %% EbbRT
    \textbf{BU Computer Science Department}, Boston, MA \hfill \textbf{January 2015 -- present} \\
    \textsl{Software Engineer \hfill \about 1 month}
    \vspace{1mm}
    \begin{itemize}[leftmargin=2em]
        \item Testing and modifying the kernel of a multi-node library runtime,
          called EbbRT, that will support exascale applications, such as those found
          in High Performance Computing centers.
    \end{itemize}
    
    %% ANDESITE
    \textbf{BU Satellite for Applications and Training}, Boston, MA \hfill \textbf{September 2014 -- present} \\
    \textsl{Software Engineer \hfill \about 5 month}
    \vspace{1mm}
    \begin{itemize}[leftmargin=2em]
        \item Developing communications and sensory software for eight picosatellites
          that each house a wireless sensor nodes (WSN). The WSN itself contains an
          Atmel ATMega 2560 chip, an RFM22B radio to send and receive data, an
          LSM9DS0 gyroscope/accelerometer/magnetometer sensor, and a global
          positioning system.
        \item Developing communications software for Boston University's ANDESITE
          nanosatellite that will receive the data that was gathered by each
          picosatellite, and send it down to the ground station using the GlobalStar Network.
        \item Creating a simulation of in-flight processes that will occur when the
          nanosatellite and picosatellites are in orbit.
        \item Developing graphical program that will display the magnitude of the magnetic
          field data detected by the LSM9DS0 module, as a function of time. 
    \end{itemize}

    %% VeSpR Post Flight Calibration
    \textbf{BU Center for Space Physics}, Boston, MA \hfill \textbf{May 2014 -- present} \\
    \textsl{Lab Assistant \hfill \about 9 months}
    \vspace{1mm}
    \begin{itemize}[leftmargin=2em]
        \item Constructing an algorithm that expresses the wavelength of light as a
          function of pixel location on the Venus Spectral Rocket (VeSpR) imager CCD.
        \item Developing software to conduct post flight analysis of the VeSpR mission result data.
    \end{itemize}
    
    %% MAVEN
    \textbf{BU Center for Space Physics}, Boston, MA \hfill \textbf{September 2014 -- October 2014} \\
    \textsl{Research Assistant \hfill \about 3 month}
    \vspace{1mm}
    \begin{itemize}[leftmargin=2em]
        \item Constructing an algorithm to automatically remove scattered background
          light in images taken by the Mars Atmosphere and Volatile Evolution
          (MAVEN) Imaging Ultraviolet Spectrograph (IUVS) instrument.
    \end{itemize}
    
    %% Vacuum Chamber Lab (Room 507)
    \textbf{BU Center for Space Physics}, Boston, MA \hfill \textbf{Summer 2014} \\
    \textsl{Lab Assistant \hfill \about 3 months}
    \vspace{1mm}
    \begin{itemize}[leftmargin=2em]
        \item Repaired electrical and mechanical defects in a damaged vacuum
          chamber that will be used for satellite systems testing.
        \item Developed software for an Arduino Mega 2560 to control four stepper motors.
    \end{itemize}
    
    
    %%__________________________________________________________________________
    %% Projects
    \section{\mysidestyle Projects}
    
    \textbf{Gabe's Status Bar \hfill September 16, 2014}
    \vspace{1mm}
    \begin{itemize}
        \item Status bar that supports custom icon widgets and event signals, meant
        to replace the non-graphical status bar that comes with Dynamic Window
        Manager (DWM).
        \item \textbf{Source: } \textsl{\href{http://github.com/gabeg805/Gabes-Status-Bar}{github.com/gabeg805/Gabes-Status-Bar}}
    \end{itemize}
    \vspace{-1mm}
    
    \textbf{Gabe's Login Manager \hfill August 7, 2014}
    \vspace{1mm}
    \begin{itemize}
        \item C based login manager I created for fun in order further customize my Linux operating
          system.
        \item \textbf{Source: } \textsl{\href{http://github.com/gabeg805/Gabes-Login-Manager}{github.com/gabeg805/Gabes-Login-Manager}}
    \end{itemize}
    \vspace{-1mm} 
    
    \textbf{USB Device Automounter \hfill April 10, 2014}
    \vspace{1mm}
    \begin{itemize}
        \item On an Arch Linux system, utilizes \textsl{systemd} and
          \textsl{systemctl} to watch the \textsl{/dev/block} directory. When a
          device is plugged into the computer, a file will be created in that
          directory that points to the device. Device information will then be
          gathered and logged, and then the device will be mounted.
        \item \textbf{Source: } \textsl{\href{http://github.com/gabeg805/Automount}{github.com/gabeg805/Automount}}
    \end{itemize}
    \vspace{-1mm}
    
    See all of my other projects on \textbf{{\href{http://github.com/gabeg805}{Github!}}}
    
    
    
    
    
    
    
    %% __________________________________________________________________________
    %% References
    %% \section{\mysidestyle References}
    %% Available upon request.
    %% \begin{tabular}{@{}p{6cm}p{6cm}}
    %% \textbf{Spencer Seal}\\
    %% Owner/CEO\\
    %% KISS Capital Corporation\\
    %% phone: (805) 705-7700\\
    %% e-mail: spencer@zazu123.com
    %% \end{tabular}
    
    %% \begin{tabular}{@{}p{6cm}p{6cm}}
    %% \textbf{Lauren Bray}\\
    %% Managing Editor\\
    %% Edhat Online Magazine\\
    %% phone: (831) 261-1103\\
    %% e-mail: lauren@edhat.com
    %% \end{tabular}
    
    
    %% ______________________________________________________________________________
\end{resume}
\end{document}

%% ______________________________________________________________________________
%% EOF
