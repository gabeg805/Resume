\documentclass[margin,line]{resume}
\usepackage[hidelinks]{hyperref}
\usepackage{cmap}
\usepackage{enumitem}
\newcommand{\about}{\raise.17ex\hbox{$\scriptstyle\mathtt{\sim}$}}

%%______________________________________________________________________________

\begin{document}
\begin{tabular*}{7in}{l@{\extracolsep{\fill}}r}
\textbf{\Large Gabriel Gonzalez}  \\
621 West Cota Unit B    &  \textbf{\href{mailto:gabeg@bu.edu}{gabeg@bu.edu}} \\
Santa Barbara, CA 93101 &  \textbf{Github:} \href{http://github.com/gabeg805}{github.com/gabeg805} \\
(805) 895-4626          &  \textbf{LinkedIn:} \href{http://linkedin.com/in/gabeg805/}{linkedin.com/in/gabeg805}
\end{tabular*}
\vspace{2mm}
\begin{resume}
    
    
    %%__________________________________________________________________________
    %% Objective / Summary
    \section{\mysidestyle Objective}
    I am looking for a software engineering internship where I can apply my knowledge
    of physics, mathematics, and programming to solve new problems and improve
    existing code. My experiences have prepared me well for learning new
    technologies and effectively contributing in a team environment.
    
    
    %%__________________________________________________________________________
    %% Computer Skills
    \section{\mysidestyle Computer\\Skills}
    \vspace{-0.01mm}
    \begin{description}
        \item[Languages:] C, IDL, Python, Java, Bash, Assembly, LaTeX
        \item[Software:] DrJava, Git, Excel, Word 
        \item[Operating systems:] Unix, Windows, Mac OS X
        \item[Courses:] Computer Systems, Data Structures and Algorithms, Electromagnetic Fields and Waves, Intermediate Mechanics, Methods of Theoretical Physics
    \end{description}
    \vspace{-3mm}
    
    
    %%__________________________________________________________________________
    %% Education
    \section{\mysidestyle Education}
    \textbf{Boston University \hfill September 2011 -- present} \\
    \textsl{BA : Astronomy and Physics}
    \vspace{1mm}
    \begin{itemize}[leftmargin=2em]
        \item \textbf{GPA:} 3.17
        \item \textbf{Expected graduation:} May 2015
    \end{itemize}
    
    
    %%__________________________________________________________________________
    %% Education
    \section{\mysidestyle Honors}
    \textbf{Boston University} \\
    \textsl{Dean's List} \hfill \textbf{Fall 2011}
    
    
    %%__________________________________________________________________________
    %% Experience
    \section{\mysidestyle Experience}
    
    %% ANDESITE
    \textbf{BU Satellite for Applications and Training}, Boston, MA \hfill \textbf{September 2014 -- present} \\
    \textsl{Software Developer \hfill \about 1 month}
    \vspace{1mm}
    \begin{itemize}[leftmargin=2em]
        \item Developing software on the Arduino Mega 2560 that controls when
        scientific data should be gathered, and then wirelessly sends the data to the
        central processing unit.
        \item Writing the drivers for the Analog to Digital Converter.
    \end{itemize}

    %% MAVEN
    \textbf{BU Center for Space Physics}, Boston, MA \hfill \textbf{September 2014 -- present} \\
    \textsl{Research Assistant \hfill \about 1 month}
    \vspace{1mm}
    \begin{itemize}[leftmargin=2em]
        \item Creating an algorithm to automatically determine the scattered light in
        an image taken by the IUVS echelle spectograph. The scattered light from the
        image will then be subtracted from the original image, resulting in an image
        with just the light from the star.
    \end{itemize}
    
    %% Vacuum Chamber Lab (Room 507)
    \textbf{BU Center for Space Physics}, Boston, MA \hfill \textbf{Summer 2014} \\
    \textsl{Lab Assistant \hfill \about 4 months}
    \vspace{1mm}
    \begin{itemize}[leftmargin=2em]
        \item Repaired mechanical and eletrical defects in a damaged vacuum
        chamber that is used for satellite systems testing.
        \item Developed software to control stepper motors using an Arduino Mega
        2560. The stepper motors control the surface inside the vacuum chamber, on
        which satellite systems may be placed and tested.
    \end{itemize}
    
    %% VeSpR Post Flight Calibration
    \textbf{BU Center for Space Physics}, Boston, MA \hfill \textbf{May 2014 -- present} \\
    \textsl{Lab Assistant \hfill \about 5 months}
    \vspace{1mm}
    \begin{itemize}[leftmargin=2em]
        \item Developing software to analyze the Venus Spectral Rocket mission result data.
        \item Post flight calibration on the Venus Spectral Rocket echelle imager and spectrograph.
    \end{itemize}
    
    %% Ionosphere of Venus and Mars
    \textbf{BU Center for Space Physics}, Boston, MA \hfill \textbf{March 2012 -- present} \\
    \textsl{Research Assistant \hfill \about 2.5 years}
    \vspace{1mm}
    \begin{itemize}[leftmargin=2em]
        \item Classifying solar flare strength and analyzing solar flare activity in
        the ionosphere of Mars using Mars Global Surveryor radio occultation data.
        \item Creating a map of the ionosphere of Venus using Pioneer Venus Orbiter
        {\it in situ} data.
        \item Determining the validity of Pioneer Venus Orbiter {\it in situ} data by
        comparing mission result data with more recent mission data.
    \end{itemize}
        
    
    %%__________________________________________________________________________
    %% Projects
    \section{\mysidestyle Projects}
    
    \textbf{Gabe's Status Bar \hfill September 16, 2014}
    \vspace{1mm}
    \begin{itemize}
        \item Status bar that supports custom icon widgets and event signals, meant
        to replace the non-graphical status bar that comes with Dynamic Window
        Manager (DWM).
        \item \textbf{Source: } \textsl{\href{http://github.com/gabeg805/Gabes-Status-Bar}{github.com/gabeg805/Gabes-Status-Bar}}
    \end{itemize}
    
    \textbf{Gabe's Login Manager \hfill August 7, 2014}
    \vspace{1mm}
    \begin{itemize}
        \item C based login manager I created for fun to customize my Linux system.
        \item \textbf{Source: } \textsl{\href{http://github.com/gabeg805/Gabes-Login-Manager}{github.com/gabeg805/Gabes-Login-Manager}}
    \end{itemize}
    \vspace{-1mm}

    \textbf{USB Device Automounter \hfill April 10, 2014}
    \vspace{1mm}
    \begin{itemize}
        \item When a USB device is plugged into a linux system, a file is created in
        \textsl{/dev} and a symbolic link is created in \textsl{/dev/block}. My program waits for a
        file to be created in \textsl{/dev/block}, mounts the device, and prints the device
        information to a log.
        \item My program is composed of several bash scripts that use native commands to Arch Linux, such as \textsl{systemd} and \textsl{systemctl}. 
        \item \textbf{Source: } \textsl{\href{http://github.com/gabeg805/Linux-Scripts/tree/master/programs/automount}{github.com/gabeg805/Linux-Scripts/tree/master/programs/automount}}
    \end{itemize}
    \vspace{-1mm}
    
    See all of my other projects on \textbf{{\href{http://github.com/gabeg805}{Github!}}}
    
    
    
    
    
    
    
    %% __________________________________________________________________________
    %% References
    %% \section{\mysidestyle References}
    %% Available upon request.
    %% \begin{tabular}{@{}p{6cm}p{6cm}}
    %% \textbf{Spencer Seal}\\
    %% Owner/CEO\\
    %% KISS Capital Corporation\\
    %% phone: (805) 705-7700\\
    %% e-mail: spencer@zazu123.com
    %% \end{tabular}
    
    %% \begin{tabular}{@{}p{6cm}p{6cm}}
    %% \textbf{Lauren Bray}\\
    %% Managing Editor\\
    %% Edhat Online Magazine\\
    %% phone: (831) 261-1103\\
    %% e-mail: lauren@edhat.com
    %% \end{tabular}
    
    
    %% ______________________________________________________________________________
\end{resume}
\end{document}

%% ______________________________________________________________________________
%% EOF
