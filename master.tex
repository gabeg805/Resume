% Packages
\documentclass{myresume}

% Define header information
\renewcommand{\Name}{Gabriel Gonzalez}
\renewcommand{\Address}{2005 Columbia Pike, Apt \#525}
\renewcommand{\City}{Arlington, VA 22204}
\renewcommand{\Phone}{(805) 895-4626}
\renewcommand{\Email}{gabeg@bu.edu}
\renewcommand{\Github}{github.com/gabeg805}
\renewcommand{\Linkedin}{linkedin.com/in/gabeg805}
\renewcommand{\Website}{gabegonzalez.me}

% ------------------------------------------------------------------------------
% Resume start
\begin{document}

  % ----------------------------------------------------------------------------
  % Populate header
  \ResumeHeader

  % ----------------------------------------------------------------------------
  % Objective
  \begin{ResumeSection}{Objective}
    \sectiontext{I am seeking a software engineering position where I can make a
      positive contribution to the technology industry by applying my knowledge of
      programming, mathematics, and astrophysics.}
  \end{ResumeSection}

  % ----------------------------------------------------------------------------
  % Computer Skills
  \begin{ResumeSection}{Computer \\ Skills}
    \explist{Languages}{C, C++, Java, Python, Bash, IDL, LaTeX, Assembly}
    \explist{Software}{Git, SVN, GNU Debugger (GDB), Valgrind}
    \explist{Operating systems}{Linux, Windows, OS X}
  \end{ResumeSection}

  % ----------------------------------------------------------------------------
  % Education
  \begin{ResumeSection}{Education}
    \university{Boston University}{May 2015}
    \degree{Bachelor of Arts}{Astronomy and Physics}
  \end{ResumeSection}

  % ----------------------------------------------------------------------------
  % Experience
  \begin{ResumeSection}{Work \\ Experience}

    % NASA
    \workplace{NASA Goddard Space Flight Center}{Greenbelt, MD}
    \workposition{Software Engineer}{June 2015 -- August 2015}

    \worktask{Designed, implemented, and tested an API to enable
      hardware-in-the-loop testing for future CubeSat missions, by integrating
      an attitude and orbit simulation program with flight software.}

    \subsectionvspace

    % EbbRT
    \workplace{Boston University Computer Science Department}{Boston, MA}
    \workposition{Software Engineer}{January 2015 -- May 2015}

    \worktask{Implemented a programmable interval timer (PIT) clock for the
      kernel of the Elastic Building block Runtime (EbbRT) research project.}
    \worktask{Completed the EbbRT clock initialization algorithm which chooses
      the correct clock (PIT or paravirtualized clock) to initialize,
      depending on the device's hardware.}

    \subsectionvspace

    % ANDESITE
    \workplace{Boston University Satellite for Applications and Training}{Boston, MA}
    \workposition{Software Engineer}{September 2014 -- May 2015}

    \worktask{Created a simplified day-in-the-life simulation for the command
      and data handling subsystem.}
    \worktask{Developed the communications algorithm between the main ANDESITE
      nanosatellite and eight picosatellites.}
    \worktask{Implemented the sensory data collection algorithm for eight picosatellites.}

    \subsectionvspace

    % VeSpR Post Flight Calibration
    \workplace{Boston University Center for Space Physics}{Boston, MA}
    \workposition{Lab Assistant}{May 2014 -- February 2015}

    \worktask{Developed an algorithm to express the wavelength of light as a
      function of pixel location on the Venus Spectral Rocket (VeSpR) imager
      CCD.}
    \worktask{Created a model of the expected spectra (1000 -- 2500\textit{\AA}) for
      Venus and Altair.}
    \worktask{Created a model for the payload quantum efficiency, reflectivity,
      and prism efficiency.}

    \subsectionvspace

    % MAVEN
    \workplace{Boston University Center for Space Physics}{Boston, MA}
    \workposition{Research Assistant}{September 2014 -- October 2014}

    \worktask{Developed an algorithm to automatically remove scattered
      background light in images taken by the Imaging Ultraviolet Spectrograph
      (IUVS) on the Mars Atmosphere and Volatile Evolution (MAVEN) space probe.}

    \subsectionvspace

    % Vacuum Chamber Lab (Room 507)
    \workplace{Boston University Center for Space Physics}{Boston, MA}
    \workposition{Lab Assistant}{June 2014 -- August 2014}

    \worktask{Developed software for a microcontroller that controls the inner
      housing inside a vacuum chamber.}

    \subsectionvspace

  \end{ResumeSection}

  % ----------------------------------------------------------------------------
  % Next page -- add resume header on the new page
  \newpage
  \ResumeHeaderReduced

  % ----------------------------------------------------------------------------
  % Experience (cont.)
  \begin{ResumeSection}{Work \\ Experience}

    % Vacuum Chamber Lab (Room 507) cont.
    \workplace{Boston University Center for Space Physics}{Boston, MA}
    \workposition{Lab Assistant}{June 2014 -- August 2014}

    \worktask{Created a CAD model of a vacuum chamber's switchboard.}
    \worktask{Tested the operating state of a damaged vacuum chamber.}
    \worktask{Replaced defective vacuum chamber equipment.}

    \subsectionvspace

    % MGS and PVO
    \workplace{Boston University Center for Space Physics}{Boston, MA}
    \workposition{Research Assistant}{February 2012 -- December 2013}

    \worktask{Developed an algorithm to analyze the response of the Mars
      ionosphere to solar flares, utilizing radio occultation measurements
      made by the Mars Global Surveyor (MGS).}
    \worktask{Categorized solar flare intensities based on the Flare Irradiance
      Spectral Model's (FISM) flare strength and the electron density
      enhancement in the M2 region of the ionosphere.}
    \worktask{Analyzed the relationship between solar flare intensity and
      Earth-Sun-Moon angle, as well as solar flare intensity and solar
      zenith angle.}
    \worktask{Created a map of the electron density in the Venus upper
      ionosphere using Pioneer Venus Orbiter (PVO) Ion Mass Spectrometer
      (OIMS) data.}
    \worktask{Compared the OIMS data to the Orbiter Electron Temperature Probe
      and Orbiter Retarding Potential Analyzer (ORPA) ionospheric data.}
    \worktask{Determined the potential invalidity of the PVO data set due to its
      skewed results when compared to VIRI and VEX.}

  \end{ResumeSection}

  % Projects
  \begin{ResumeSection}{Projects}
    \projectname{Elysia}
    \projecttype{Login Manager}
    \projectdesc{A GTK+ based login manager written entirely in C that is
      designed to be customizable to the user's preference and have completely
      open source code.}

    \subsectionvspace

    \projectname{Atlas}
    \projecttype{Status Bar}
    \projectdesc{A gtkmm based C++ status bar designed to be a replacement for
      the dwm text status bar. It is an improvement to the default dwm status
      bar in that it allows for both text and images to be displayed anywhere on
      the status bar, and additionally allows for the status bar itself to be
      placed in any orientation and location on the X server.}

    \subsectionvspace

    \projectname{Aria}
    \projecttype{Notification Bubble}
    \projectdesc{A gtkmm based C++ notification bubble that is designed to be
      highly configurable, with the user able to customize the location it is
      displayed, font, text size, background color, foreground color, and more.}
  \end{ResumeSection}

\end{document}
