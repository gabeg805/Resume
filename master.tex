%% Packages
\documentclass[10pt]{article}
\usepackage[hidelinks]{hyperref}
\usepackage{cmap}
\usepackage{enumitem}
\usepackage{amsmath}
\usepackage[top=0.5in, bottom=0.5in, left=0.5in, right=0.5in]{geometry}
\usepackage{array}
\usepackage{microtype}
%% \usepackage{showframe}
%% \usepackage{layout}

%% Header/footer margins 
\topmargin=-1in
\headheight=0pt
\headsep=0.5in
\footskip=0pt
\marginparwidth=0pt
\marginparpush=0pt

% Itemize spacing
%% \setlist{nolistsep}
%% \setlist{noitemsep}
%% \setlist{nosep}
\setlist{leftmargin=1em}
\setlist[itemize,1]{label=$\circ$}

%% Resume info (fields can be empty)
\newcommand{\Name}{Gabriel Gonzalez}
\newcommand{\Address}{621 West Cota Street Unit B}
\newcommand{\City}{Santa Barbara, Ca 93101}
\newcommand{\Phone}{(805) 895-4626}
\newcommand{\Email}{gabeg@bu.edu}
\newcommand{\Github}{github.com/gabeg805}
\newcommand{\Linkedin}{linkedin.com/in/gabeg805}
\newcommand{\Website}{gabegonzalez.me}

% Environments
\newenvironment{ResumeSection}[1]{
  \begin{tabular}{ p{0.1\textwidth} p{0.8\textwidth} }
    \SectionTitle{\SectionTitleStack{#1}}
}{
  \SectionEnd
  \end{tabular}
}

\newenvironment{ResumeWorkSection}[1]{
  \begin{tabular}{ p{0.1\textwidth} p{0.8\textwidth} }
    \SectionTitle{\SectionTitleStack{#1}}
}{
  \WorkEnd
  \end{tabular}
}

\newenvironment{WorkItemize}{
  \begin{minipage}[t]{0.97\linewidth}
    \begin{itemize}[parsep=0.125em]
}{
    \end{itemize}
  \end{minipage}
}

%% Resume section styling
\newcommand{\FontSize}{10pt}
\newcommand{\SectionTitleStack}[1]{\smash[b]{\begin{tabular}[t]{@{}c@{}}#1\end{tabular}}}
\newcommand{\SectionTitle}[1]{\textsc{\small #1}}
\newcommand{\SectionItem}[1]{   &         #1              \vspace{0.25em} \\}
\newcommand{\SectionItemB}[1]{  & \textbf{#1}             \vspace{0.25em} \\}
\newcommand{\SectionItemI}[1]{  & \textsl{#1}             \vspace{0.25em} \\}
\newcommand{\SectionItemBR}[2]{ & \textbf{#1}         #2  \vspace{0.25em} \\}
\newcommand{\SectionItemBI}[2]{ & \textbf{#1} \textsl{#2} \vspace{0.25em} \\}
\newcommand{\SectionEnd}{\vspace{1em} \\}

%% Resume employment styling
\newcommand{\WorkEmployer}[1]{ & \textbf{#1} \hfill}
\newcommand{\WorkLocation}[1]{   \textbf{#1} \vspace{0.25em} \\}
\newcommand{\WorkPosition}[1]{ & \textsl{#1} \hfill}
\newcommand{\WorkDate}[1]{       \textsl{#1} \vspace{0.5em} \\}
\newcommand{\WorkItem}[1]{& \begin{WorkItemize} \item #1 \end{WorkItemize} \\}
\newcommand{\WorkEnd}{\vspace{1em} \\}

%% Header
\newcommand{\ResumeHeader}{
  \begin{tabular*}{7in}{l@{\extracolsep{\fill}}r}
    \textbf{\Large \Name} & \textbf{\href{mailto:\Email}{\Email}} \\
    \Address              & \textbf{Website:} \href{http://\Website}{\Website} \\
    \City                 & \textbf{Github:}  \href{http://\Github}{\Github} \\
    \Phone                & \textbf{LinkedIn:} \href{http://\Linkedin}{\Linkedin} \\
  \end{tabular*}

  \hspace{2mm}\rule{0.92\textwidth}{0.4pt}
  \vspace{1.2em}
}

\newcommand{\ResumeHeaderReduced}{
  \begin{tabular*}{7in}{l@{\extracolsep{\fill}}r}
    \textbf{\large \Name} & \textbf{\href{mailto:\Email}{\Email}} \\
  \end{tabular*}

  \vspace{-0.5em}
  \hspace{2mm}\rule{0.92\textwidth}{0.4pt} 
  \vspace{1.2em}
}


%% Resume start
%% \pagenumbering{gobble}
%% \thispagestyle{empty}
\begin{document} %\layout
\ResumeHeader

  %% Objective
  \begin{ResumeSection}{Objective}
    \SectionItem{I am seeking a software engineering position where I can make a
      positive contribution to the technology industry by applying my knowledge of
      programming, mathematics, and astrophysics.
    }
  \end{ResumeSection}

  %% Computer Skills
  \begin{ResumeSection}{Computer \\ Skills}
    \SectionItemBR{Languages:}{C, C++, Java, Python, Bash, IDL, LaTeX, Assembly}
    \SectionItemBR{Software:}{Git, SVN, GNU Debugger (GDB), Valgrind}
    \SectionItemBR{Operating systems:}{Unix, Windows, OS X}
  \end{ResumeSection}

  %% Education
  \begin{ResumeSection}{Education}
    \SectionItemB{Boston University \hfill May 2015}
    \SectionItemI{BA : Astronomy and Physics}
  \end{ResumeSection}

  %% Experience
  \begin{ResumeWorkSection}{Work \\ Experience}
    %% NASA
    \WorkEmployer{NASA Goddard Space Flight Center}
    \WorkLocation{Greenbelt, MD}
    \WorkPosition{Software Engineer}
    \WorkDate{June 2015 -- present}
    \WorkItem{Integrating an orbit simulation program, 42, with flight software
      in order to enable CubeSat Hardware-in-the-Loop testing.}
  \end{ResumeWorkSection}

  %% EbbRT
  \begin{ResumeWorkSection}{}
    \WorkEmployer{Boston University Computer Science Department}
    \WorkLocation{Boston, MA}
    \WorkPosition{Software Engineer}
    \WorkDate{January 2015 -- May 2015}
    & \begin{WorkItemize}
        \item Implemented a programmable interval timer (PIT) clock for the
          kernel of the Elastic Building block Runtime (EbbRT), a multi-node
          library runtime.
        \item Completed the EbbRT clock interface which initializes the correct
          clock (PIT or paravirtualized clock), depending on the device's
          hardware.
      \end{WorkItemize}
  \end{ResumeWorkSection}

  %% ANDESITE
  \begin{ResumeWorkSection}{}
    \WorkEmployer{Boston University Satellite for Applications and Training}
    \WorkLocation{Boston, MA}
    \WorkPosition{Software Engineer}
    \WorkDate{September 2014 -- May 2015}
    & \begin{WorkItemize}
        \item Created a simplified day-in-the-life simulation for the command
          and data handling subsystem.
        \item Developed the communications protocol between the main ANDESITE
          nanosatellite and eight picosatellites.
        \item Implemented the sensory data collection algorithm for eight picosatellites.
      \end{WorkItemize}
  \end{ResumeWorkSection}

  %% VeSpR Post Flight Calibration
  \begin{ResumeWorkSection}{}
    \WorkEmployer{Boston University Center for Space Physics}
    \WorkLocation{Boston, MA}
    \WorkPosition{Lab Assistant}
    \WorkDate{May 2014 -- February 2015}
    & \begin{WorkItemize}
        \item Developed an algorithm to express the wavelength of light as a
          function of pixel location on the Venus Spectral Rocket (VeSpR) imager
          CCD.
        \item Created a model of the expected spectra (1000 -- 2500\textit{\AA}) for
          Venus and Altair.
        \item Created a model for the payload quantum efficiency, reflectivity,
          and prism efficiency.
      \end{WorkItemize}
  \end{ResumeWorkSection}

  %% MAVEN
  \begin{ResumeWorkSection}{}
    \WorkEmployer{Boston University Center for Space Physics}
    \WorkLocation{Boston, MA}
    \WorkPosition{Research Assistant}
    \WorkDate{September 2014 -- October 2014}
    \WorkItem{Developed an algorithm to automatically remove scattered
      background light in images taken by the Mars Atmosphere and Volatile
      Evolution (MAVEN) Imaging Ultraviolet Spectrograph (IUVS) instrument.
    }
  \end{ResumeWorkSection}

  %% Vacuum Chamber Lab (Room 507)
  \begin{ResumeWorkSection}{}
    \WorkEmployer{Boston University Center for Space Physics}
    \WorkLocation{Boston, MA}
    \WorkPosition{Lab Assistant}
    \WorkDate{June 2014 -- August 2014}
    & \begin{WorkItemize}
        \item Developed software for a microcontroller to control the inner
          housing that holds items placed in the vacuum chamber.
      \end{WorkItemize}
  \end{ResumeWorkSection}

  %% Add resume header for next page
  \vspace{1em}
  \ResumeHeaderReduced

  %% Vacuum Chamber Lab (Room 507) cont.
  \begin{ResumeWorkSection}{Work \\ Experience}
    \WorkEmployer{Boston University Center for Space Physics}
    \WorkLocation{Boston, MA}
    \WorkPosition{Lab Assistant}
    \WorkDate{June 2014 -- August 2014}
    & \begin{WorkItemize}
        \item Created a CAD model of the vacuum chamber's switchboard.
        \item Tested the operating state of the damaged vacuum chamber.
        \item Replaced defective vacuum chamber equipment.
      \end{WorkItemize}
  \end{ResumeWorkSection}

  %% MGS and PVO
  \begin{ResumeWorkSection}{}
    \WorkEmployer{Boston University Center for Space Physics}
    \WorkLocation{Boston, MA}
    \WorkPosition{Research Assistant}
    \WorkDate{February 2012 -- December 2013}
    & \begin{WorkItemize}
        \item Developed an algorithm to analyze the response of the Mars
          ionosphere to solar flares, utilizing radio occultation measurements
          made by the Mars Global Surveyor (MGS).
        \item Categorized solar flare intensities based on the Flare Irradiance
          Spectral Model's (FISM) flare strength and the electron density
          enhancement in the M2 region of the ionosphere.
        \item Analyzed the relationship between solar flare intensity and
          Earth-Sun-Moon angle, as well as solar flare intensity and solar
          zenith angle.
        \item Created a map of the electron density in the Venus upper
          ionosphere using Pioneer Venus Orbiter (PVO) Ion Mass Spectrometer
          (OIMS) data.
        \item Compared the OIMS data to the Orbiter Electron Temperature Probe
          and Orbiter Retarding Potential Analyzer (ORPA) ionospheric data.
        \item Determined the potential invalidity of the PVO data set due to its
          skewed results when compared to VIRI and VEX.
      \end{WorkItemize}
  \end{ResumeWorkSection}

  %% Projects
  \vspace{0.5em}
  \begin{ResumeSection}{Projects}
    \SectionItemB{Elysia}
    \SectionItemI{Login Manager}
    \SectionItem{A GTK+ based login manager written entirely in C that is
      designed to be customizable to the user's preference and completely
      transparent.
    }
    \SectionEnd

    \SectionItemB{Atlas}
    \SectionItemI{Status Bar}
    \SectionItem{A gtkmm based C++ status bar designed to be a replacement for
      the dwm text status bar. It is an improvement to the default dwm status
      bar in that it allows for both text and images to be displayed anywhere on
      the status bar, and additionally allows for the status bar itself to be
      placed in any orientation and location on the X server.
    }
    \SectionEnd

    \SectionItemB{Aria}
    \SectionItemI{Notification Bubble}
    \SectionItem{A gtkmm based C++ notification bubble that is designed to be
      highly configurable, with the user able to customize the location it is
      displayed, font, text size, background color, foreground color, and more.
    }
  \end{ResumeSection}

\end{document}
